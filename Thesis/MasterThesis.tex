\documentclass[11pt]{article}

\usepackage[english]{babel}
\usepackage[utf8]{inputenc}
\usepackage{a4wide}
\usepackage{graphicx}
\usepackage{wrapfig}
\usepackage{subcaption}
\usepackage[margin=1.5cm]{caption}
%\usepackage[framed,numbered]{mcode}
\usepackage{amsmath}
%\usepackage{arevmath}
\usepackage{amssymb}
\usepackage{amsthm}
\usepackage{physics}
\usepackage[version=4]{mhchem}
\usepackage{array}
\usepackage{cases}
\usepackage{pbox}
\usepackage{hepnames}
\usepackage{siunitx}
\usepackage{enumitem}
\usepackage{fancyhdr}
\usepackage{etoolbox}

\patchcmd{\thebibliography}{\section*{\refname}}{}{}{}

\setlength{\parindent}{0.3cm}
\addtolength{\oddsidemargin}{-5mm}
\addtolength{\evensidemargin}{-5mm}
\addtolength{\textwidth}{10mm}
\setlength{\headheight}{14pt}

\pagestyle{fancy}

\begin{document}
	\begin{titlepage}
		\begin{center}
			\hrule
			\vspace*{1cm}
			
			\textbf{\LARGE{Flashlamp Annealing for Improved Ferroelectric Junctions}}\\ 
			\vspace{0.5cm}
			\Large{Master's Degree Project}
			
			\vspace{1.5cm}
			
			\textbf{\Large{Theodor Blom}} \\
			\small{Lund University} \\
			\vspace{0.2cm}
            \small{Project Duration: 12 months, 60 hp}\\
            \vspace{0.2cm}
			\small{\today}
			
			\vfill
			
			\includegraphics[width=0.35\textwidth]{~/Pictures/lu_logo-portrait-en.png}
			
			\vspace{1cm}

            \begin{minipage}{0.51\textwidth}
                \centering
                \textbf{\Large{Mattias Borg}} \\
			    \vspace{0.2cm}
                Division of Nano Electronics,\\
                Department of Electrical and Informations Technology,\\
                Faculty of Engineering, LTH,\\
			    Lund University\\
            \end{minipage}
            \begin{minipage}{0.48\textwidth}
                \centering
                \textbf{\Large{Rainer Timm}} \\
			    \vspace{0.2cm}
                Division of Synchrotron Radiation,\\
                Department of Physics,\\
                Faculty of Science,\\
			    Lund University\\
            \end{minipage}
			\vspace{1cm}
			\hrule
		\end{center}
	\end{titlepage}
\newpage
    \thispagestyle{empty}
    \mbox{}
\newpage
    \tableofcontents
    \begin{abstract}
        Abstract here!
    \end{abstract}
\newpage \pagenumbering{arabic}
    \section{Introduction}

    Mål: Introducera området och ge en överblick.\cite{atle2019development}

    \section{Semiconductors and Ferroelectrics}

    Mål: Klargöra varför III-V (utgå från Si) och FE är intressant. Varför gör vi detta? Vad är applikationerna!

        \subsection{III-V Semiconductors}

        Mål: Redogör för varför III-V är intressant.

        \subsection{Ferroelectricity}

        Mål: Basics of FE;\ Kristallstrukturer, Polarisation, Domäner och PE-kurvor.

        \subsection{\ce{HfZrO2}}

        Mål: Redogör för FE-\ce{HfO2} och beskriv hur \ce{Zr} kommer in i bilden.

            \subsubsection{TiN-capping?}

            Mål: Ta upp varför vi använder TiN. Ta upp W?\ 

        \subsection{Energyband Theory and Leakage Mechanics}

        Mål: Redogör för hur energibanden ser ut med Semiconductor-Insulator-Metal-cap och gå igenom de olika tunnelsätten.

        \subsection{FTJs}

        Mål: Sammanställ allt genom att beskriva hur FTJer funkar och fördelarna med dem.

    \section{Fabrication}

    Mål: Redogör för hela processen på LNL.\ 

        \subsection{Processing Methods}

        Mål: Redogör för dem mest intressanta/relevanta metoderna. Kanske bara ALD och FLA?\ 

            \subsubsection{ALD}


            \subsubsection{Sputtering}


            \subsubsection{Flashlamp Annealing}


            \subsubsection{Thermal Evaporation}


        \subsection{Sample Fabrication Process}

        Mål: Redogör för hela min process.

    \section{Electrical Charcterization}

    Mål: Redogör för metoderna på E-huset.

        \subsection{PUND and Endurance}


        \subsection{CV}


    \section{Results and Analysis}

    Mål: Presentera serierna i en rimlig ordning och dra slutsatser från varje serie.

    Tre steg: 1. FlashInt och Temperatur 2. FlashNum vid olika FlashInt 3. Utforska olika kompositioner (laminat och superlattice)

    Fokus på frågeställingar!
    \begin{itemize}
        \item Utforska möjligheten kring FLA
        \item Hur bra devices kan vi göra? PUND och Endurance
        \item Fokus på interface defects genom CV!\  
    \end{itemize}

    \section{Conclussion}

    Mål: Wrap it up. Lägg fram de främsta resultaten/ideerna och ge tips på hur man kan undersöka vidare.

    \section{References}
        \bibliography{MasterThesis.bib}
        \bibliographystyle{ieeetr}

\end{document}
